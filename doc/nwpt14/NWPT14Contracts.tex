\documentclass[a4paper,debug,twocolumn]{easychair}
%\documentclass[a4paper,debug]{easychair}

\usepackage[utf8]{inputenc}
\usepackage[english]{babel}
\usepackage{amsmath}
\usepackage{amsfonts}
\usepackage{amssymb}
\usepackage{stmaryrd}
%\usepackage{graphicx}
\usepackage{xcolor}

\newcommand{\comm}[3][red]{{\small \color{#1}{$\spadesuit$#2: #3}}}
%\renewcommand{\comm}[3][]{}
\newcommand{\jbcomment}[1]{\comm[orange]{jb}{#1}}
\newcommand{\pbcomment}[1]{\comm[green]{pb}{#1}}
\newcommand{\mecomment}[1]{\comm[magenta]{me}{#1}}

\newcommand\type[1]{\mathsf{#1}}
\newcommand\reals{{\mathbb R}}
\newcommand\nats{{\mathbb N}}
\newcommand\ints{{\mathbb Z}}
\newcommand\pto{\rightharpoonup}
\newcommand\cSem[2]{{\mathcal C}\left\llbracket#1\right\rrbracket_{#2}}

\newcommand\cRed[2]{\stackrel{#2}\Rightarrow_{#1}}
\newcommand\cRedFun{f_{\Rightarrow}}
\newcommand\envAdv[1]{#1\uparrow}


\theoremstyle{plain} 
\newtheorem{theorem}{Theorem} 

\begin{document}

\title{Towards Certified Management of Financial Contracts
\thanks{Work partially supported by DSF grant No.\jbcomment{yaddayadda} (\textsc{Hiperfit})}}

\titlerunning{Towards Certified Management of Financial Contracts}

\author{Patrick Bahr
    \and
        Jost Berthold 
    \and 
        Martin Elsman 
    \and 
        Fritz Henglein\\
}

\institute{University of Copenhagen\\
    Dept. of Computer Science (DIKU)\\
    \email{\{paba,berthold,mael,henglein\}@di.ku.dk}\\
}

\authorrunning{Bahr, Berthold, Elsman, Henglein}

\clearpage
\maketitle

\section{Introduction}

Financial computations are among the domains where (embedded) Domain specific
languages (EDSLs) first came into widespread use.
The seminal work on modelling derivatives by Peyton-Jones and Eber~\cite{SPJ2000}
demonstrated the usefulness of DSLs for financial software.
%
Eber's company \emph{LexiFi}, a leading software provider
for financial institutions, has designed their entire software portfolio
around contract language MLFi~\cite{MLFi} (ML for finance).

What was a complete novelty in 2000 has become common-place since.
Today, many banks and financial software providers use contract DSLs internally
which follow their first design to some extent.
However, the in-house languages we have encountered within our work in project
\textsc{Hiperfit}~\cite{TFP11Hiperfit} are sometimes disappointing with respect
to what could be achieved.
%
Some contract languages are used to ease pay-out batch processing and suffer
from unfavourable data models when used for pricing.
%
Other languages are used mainly for pricing and valuation, and contain many
bolted-on extensions for bespoke stochastic models and in-house pricing
software, which get in the way of readability and symbolic contract processing.
\jbcomment{tone down? how to substantiate it?}


In this paper, we present a small simple contract language which rigorously
relegates any artefacts of modelling and computation from its core, and show
that plenty of information can be derived from, and useful manipulations defined
on, just the symbolic contract specification, independent of any stochastic
aspects of the modelled contracts.
Contracts modelled in our language can be analysed and transformed for their
management according to a precise cash-flow semantics, modelled and checked
using proof assistant Coq.

\section{Contract Language}
\label{sec:contract-language}

\subsection{Example}
\label{sec:example}

\subsection{Semantics}
\label{sec:semantics}

We define a denotational semantics by mapping contracts to cash flows,
which are partial functions from time to transfers between two
parties:
\begin{align*}
  \type{Trans} &= \type{Party} \times \type{Party} \times
  \type{Currency} \to \reals\\
  \type{Flow} &= \nats \pto Trans
\end{align*}

The cash flow is a partial mapping since it may not be determined due
to insufficient knowledge of the values of observables. The semantics
of contracts depends on the valuation of observables, which is given
by an environment:
\begin{align*}
  \type{Env} &= \ints \times \type{Observable} \pto A\\
  \cSem{\cdot}{\cdot}&\colon \type{Contract} \times \type{Env} \to
  \type{Flow}
\end{align*}

Secondly, we define an operational semantics in the form of a
reduction semantics. This reduction semantics advances a contract by
one time unit. We write $c \cRed \rho \tau c'$, to denote that $c$ is
advanced to $c'$ in the environment $\rho$, where $\tau \in
\type{Trans}$ indicates the transfers that are necessary (and
sufficient) in order to advance $c$ to $c'$.

The reduction semantics can be implemented as a recursive function of
type 
\[
\cRedFun \colon\type{Contract}\times \type{Env} \pto \type{Contract} \times \type{Trans}
\]
$\cRedFun$ takes a contract $c$ and an environment $\rho$, and returns
the residual contract $c'$ and the transfers $\tau$ such that $c
\cRed\rho\tau c'$. The argument $\rho$ typically contains the
knowledge that we have about the observables up to the present time,
i.e.\ for time points $\leq 0$.

We can show that the operational semantics is sound w.r.t.\ the
denotational semantics:
\begin{theorem}
  Let $c \in \type {Contract}$ and $\rho,\rho' \in \type{Env}$ such
  that $\rho'(i,o) = \rho(i,o)$ for all $(i,o) \in
  \mathsf{dom}(\rho)$.
  \begin{enumerate}
  \item If $c \cRed\rho\tau c'$, then $\cSem c {\rho'} (0) = \tau$ and
    $\cSem c {\rho'} (i+1) = \cSem{c'}{\envAdv{\rho'}}(i)$ for all $i
    \in \nats$, where $\envAdv{\rho'} (i) = \rho'(i+1)$
  \item If there are no $c', \tau$ with $c \cRed\rho\tau c'$, then
    $\cSem c \rho (0)$ is undefined.
  \end{enumerate}
\end{theorem}


\subsection{Contract Analysis}
\label{sec:contract-analysis}

\begin{itemize}
\item Causality
\item Horizon
\item Depends on (what observables does the contract depend on)
\item Triggers
\end{itemize}


\section{Contract Management and Transformation}
\label{sec:contract-management}

What do we want to do with contracts?

- Transform contracts
- Advancing a contract (reduction semantics of contracts).

\subsection{Formal Treatment}
\label{sec:properties}

\begin{itemize}
\item Soundness of reduction semantics
\item Correctness of contract equivalences
\item Correctness of contract analysis functions
\end{itemize}


\subsection{Implementation}
\label{sec:implementation}

Implementation as a Haskell EDSL

\section{Discussion and Future Work}
\label{sec:disc-future-work}

Future work:
\begin{itemize}
\item more extensive analyses (e.g. ``zooming'', scenario generation)
\item code generation for pay-off function for pricing frameworks
\end{itemize}





\bibliographystyle{abbrv}
\bibliography{NWPT14Contracts}

\end{document}