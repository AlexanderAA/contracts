\documentclass[a4paper,debug,twocolumn]{easychair}
%\documentclass[a4paper,debug]{easychair}

\usepackage[utf8]{inputenc}
\usepackage[english]{babel}
\usepackage{amsmath}
\usepackage{amsfonts}
\usepackage{amssymb}
\usepackage{stmaryrd}
%\usepackage{graphicx}
\usepackage{xcolor}

\newcommand{\comm}[3][red]{{\small \color{#1}{$\spadesuit$#2: #3}}}
%\renewcommand{\comm}[3][]{}
\newcommand{\jbcomment}[1]{\comm[orange]{jb}{#1}}
\newcommand{\pbcomment}[1]{\comm[green]{pb}{#1}}
\newcommand{\mecomment}[1]{\comm[magenta]{me}{#1}}

\newcommand\type[1]{\mathsf{#1}}
\newcommand\reals{{\mathbb R}}
\newcommand\nats{{\mathbb N}}
\newcommand\ints{{\mathbb Z}}
\newcommand\bools{{\mathbb B}}
\newcommand\pto{\rightharpoonup}
\newcommand\cSem[2]{{\mathcal C}\left\llbracket#1\right\rrbracket_{#2}}

\newcommand\cRed[2]{\stackrel{#2}\Rightarrow_{#1}}
\newcommand\cRedFun{f_{\Rightarrow}}
\newcommand\envAdv[1]{#1\uparrow}
\newcommand\Pred[1]{\textsc{#1}}

\theoremstyle{plain} 
\newtheorem{theorem}{Theorem} 

\begin{document}

\title{Towards Certified Management of Financial Contracts\thanks{This
    work has been partially supported by the Danish Council for
    Strategic Research under contract number 10-092299
    (\textsc{Hiperfit}~\cite{TFP11Hiperfit}), and the Danish Council
    for Independent Research under Project 12-132365.}}

\titlerunning{Towards Certified Management of Financial Contracts}

\author{Patrick Bahr
    \and
        Jost Berthold
    \and 
        Martin Elsman\\
}

\institute{University of Copenhagen\\
    Dept. of Computer Science (DIKU)\\
    \email{\{paba,berthold,mael\}@di.ku.dk}\\
}

\authorrunning{Bahr, Berthold, Elsman}

\clearpage
\maketitle

\section{Introduction}

Banks and financial institutions nowadays often use domain-specific
languages (DSLs) for describing complex financial contracts, in
particular, for specifying how asset transfers for a specific contract
depend on underlying observables, such as interest rates, currency
rates, and stock prices.

The seminal work by Peyton-Jones and Eber on financial
contracts~\cite{SPJ2000} shows how an algebraic approach to contract
specification can be used for valuation of contracts (when combined
with a model of the underlying observables) and for managing how
contracts evolve under so-called fixings\footnote{Underlying
  observables gradually become fixed when time passes by.} and
decision-taking, with the contracts eventually evaporating into the
empty contract, for which no party have further obligations. The ideas
have emerged into Eber's company \emph{LexiFi}, which has become a
leading software provider for a range of financial institutions, with
all contract management operations centralised around a
domain-specific contract language hosted in MLFi~\cite{MLFi}, a
derivative of the functional programming language OCaml.

In this paper, we present a small simple contract language, which
rigorously relegates any artefacts of modelling and computation from
its core. The language shares the same vision as the previously
mentioned work with the addition that it (a) allows for specifying
multi-party contracts (such as entire portfolios), (b) has good
algebraic properties, well suited for formal reasoning, and yet (c)
allows for expressing many interesting contracts appearing in
real-world portfolios, such as various variations of so-called barrier
options.

We show that plenty of information can be derived from, and useful
manipulations defined on, just the symbolic contract specification,
independent of any stochastic aspects of the modelled contracts.
Contracts modelled in our language are analysed and transformed for
management according to a precise cash-flow semantics, modelled and
checked using the Coq proof assistant.

Implementations of the contract language in Haskell and Coq are
available online\footnote{See
  \url{https://github.com/HIPERFIT/contracts}.} together with
machine-checkable proofs (in Coq) of the key properties of the
contract language.

\section{Contract Language}
\label{sec:contract-language}

\subsection{Language Constructs}
\label{sec:language-constructs}

Financial contracts essentially define future transactions (\emph{cash-flows})
between different parties who agree on a contract.
Amounts in contracts may be \emph{scaled} using real-valued expression
($\type{Expr}_\reals$), which may refer to \emph{observable} underlying
values, such as foreign exchange rates, stock prices, or market indexes.
Likewise, contracts can contain \emph{alternatives} depending on Boolean
predicates ($\type{Expr}_\bools$), which may refer to these observables,
as well as external \emph{decisions} taken by the parties involved.

Observables and choices in our contract language are ``observed'' with
a given offset from the current day.
In general, all definitions use \emph{relative time},
aiding the compositionality of contracts.  

A common contract structure is to \emph{repeat} a choice
between alternatives until a given end date.
As a simple example, consider an \emph{FX option} on US dollars:
Party X may, within 90 days, decide whether to buy $100$ US dollars for
a fixed rate $r$ of Danish Kroner from Party Y.

\vspace*{-2ex}
{\footnotesize
\begin{align*}
\mathit{option} &= \mathit{checkWithin}(\mathit{chosenBy}(X,0), 90, \mathit{trade}, \mathit{zero})\\
\mathit{trade} &= \mathit{scale}(100, both(\mathit{transfer}(Y, X,\type{USD}), pay))\\
\mathit{pay} &= \mathit{scale}(r, \mathit{transfer}(X, Y,\type{DKK}))
\end{align*}

%\vspace*{-2ex}
}
%
%More complex examples are given in the appendix, using our
%implementation of the contract language as a Haskell EDSL.

The \emph{checkWithin} construct which generalises an alternative by
iterating the decision (\emph{chosenBy}) of party X.
If X chooses at one day before the end (90 days), the \emph{trade} comes
into effect, consisting of two transfers (\emph{both}) between the parties.
Otherwise, the contract becomes empty (\emph{zero}) after 90 days.

The contracts atoms and combinators of the language are:

%\jbcomment{or in the Appendix?}
\vspace*{-3ex}
{\small

\begin{align*}
\mathit{zero} &: \type{Contr}\\[-0.6ex]
\mathit{transfer} &: \type{Party} \times \type{Party} \times \type{Currency}\to \type{Contr}\\[-0.6ex]
\mathit{scale}&: \type{Expr}_\reals \times \type{Contr} \to \type{Contr}\\[-0.6ex]
\mathit{translate}&: \nats \times \type{Contr} \to \type{Contr}\\[-0.6ex]
\mathit{checkWithin}&:  \type{Expr}_\bools \times \nats \times \type{Contr} \times \type{Contr} %\\&
\to \type{Contr}\\[-0.6ex]
\mathit{both}&: \type{Contr} \times \type{Contr} \to \type{Contr}
\end{align*}
\vspace*{-3ex}
}

\emph{translate(n)} simply translates a contract $n$ days into the future.

%We present foreign exchange (FX) and stock market
%derivatives, but the concepts generalise easily. Likewise, cash-flows are
%based on a fixed set of \emph{currencies}, where they could instead allow for
%arbitrary assets.
%\jbcomment{Currency is a closed type, while observables are open (just strings)}

In the expression language, we also define a special
expression $\mathit{acc}$ which \emph{accumulates} a value over a
given number of days from today.

\vspace*{-2ex}
{\small
\[
\mathit{acc} : (\type{Expr}_\alpha \to \type{Expr}_\alpha) \to \nats \to \type{Expr}_\alpha \to \type{Expr}_\alpha
\]
}
\vspace*{-3ex}

The accumulator can be used to compute averages (for so-called Asian options),
or more generally to carry forward a state while computing values.
%\jbcomment{explain: general -- carries state, for instance computing averages}


\subsection{Denotational Semantics}
\label{sec:semantics}

The semantics of a contract is given by its cash-flow, which is a
partial mapping from time to transfers between two parties:
\begin{align*}
  \type{Trans} &= \type{Party} \times \type{Party} \times
  \type{Currency} \to \reals\\
  \type{Flow} &= \nats \pto \type{Trans}
\end{align*}

The cash-flow is a partial mapping since it may not be determined due
to insufficient knowledge about observables and external decisions,
provided by an environment $\rho\in\type{Env}$:
\begin{align*}
  \type{Env} = &\ \ints \to X \pto  \reals \cup \bools\\
  \cSem{\cdot}{\cdot}&\colon \type{Contr} \times \type{Env} \to
  \type{Flow}
\end{align*}
Note that the environment is a partial mapping from $\ints$, i.e.\ it
may provide information about the past.

This denotational semantics is the foundation for the formalisation of
symbolic contract analyses, contract management and %contract
transformations.

An important property of the semantics of contracts is monotonicity,
i.e.
\[
\cSem c {\rho_1} \subseteq \cSem c {\rho_2} \text{ if }
\rho_1\subseteq \rho_2
\]
where $\subseteq$ denotes the subset inclusion of the graph of two
partial functions.

\subsection{Contract Analysis}
\label{sec:contract-analysis}

When dealing with contracts we are interested in a number of semantic
properties of contracts, e.g.\ causality (Does the cash-flow at each
time $t$ depend only on observables at time $\leq t$?), horizon (From
which time onwards is the cash-flow always zero?) and dependencies
(Which observables does the cash-flow depend on?). Such properties can
be characterised precisely using the denotational semantics. For
example a contract $c$ is causal iff for all $t \in \nats$ and
$\rho_1, \rho_2 \in \type{Env}$ such that $s \le t$ implies
$\rho_1(s)=\rho_2(s)$ for all $s \in \ints$, we have that $\cSem
c{\rho_1} (t) = \cSem c {\rho_2} (t)$. That is, the cash-flows at any
time $t$ do not depend on observables and decisions after $t$.

It is in general undecidable whether a contract is causal, but we
can provide conservative approximations.  For instance we have an
inductively defined predicate $\Pred{Causal}$ such that if
$\Pred{Causal}(c)$, then $c$ is indeed causal. This is not unlike type
checking, which provides a conservative approximation of type safety.


\section{Contract Management and Transformation}
\label{sec:contract-management}

Apart from a variety of analyses our framework provides functionality
to transform contracts in meaningful ways. The most basic form of such
transformations are provided by algebraic laws. These laws of the form
$c_1 \equiv c_2$ state when it is safe to replace a contract $c_1$ by
an equivalent contract $c_2$. Using our denotational semantics, these
algebraic laws can be proved in a straightforward manner: we have $c_1
\equiv c_2$ iff $\cSem{c_1}\rho = \cSem{c_2}\rho$ for all $\rho \in
\type{Env}$.

More interesting are transformations that are based on knowledge about
observables and external decisions. That is, we transform a contract
$c$ based on an environment $\rho \in \type{Env}$ that encodes the
knowledge that we already have. We consider two examples,
specialisation and reduction.

\subsection{Specialisation}
\label{sec:specialisation}

A specialisation function $f$ performs a partial evaluation of a
contract $c$ under a given environment $\rho$. The resulting contract
$f(c,\rho)$ is equivalent to $c$ under the environment $\rho$. More
generally, we have that $\cSem{f(c,\rho)}{\rho'} = \cSem c \rho$ for
any environment $\rho' \subseteq \rho$, including the empty
environment.



\subsection{Reduction Semantics}
\label{sec:reduction-semantics}


Apart from the denotational semantics our contract language is also
equipped with a reduction semantics~\cite{andersen06sttt}, which
advances a contract by one time unit. We write $c \cRed \rho \tau c'$,
to denote that $c$ is advanced to $c'$ in the environment $\rho$,
where $\tau \in \type{Trans}$ indicates the transfers that are
necessary (and sufficient) in order to advance $c$ to $c'$.

The reduction semantics can be implemented as a recursive function of
type 
\[
\cRedFun \colon\type{Contr}\times \type{Env} \pto \type{Contr} \times \type{Trans}
\]
$\cRedFun$ takes a contract $c$ and an environment $\rho$, and returns
the residual contract $c'$ and the transfers $\tau$ such that $c
\cRed\rho\tau c'$. The argument $\rho$ typically contains the
knowledge that we have about the observables up to the present time,
i.e.\ for time points $\leq 0$.

We can show that the reduction semantics is sound and complete w.r.t.\
the denotational semantics:
\begin{theorem}
  If $c \cRed\rho\tau c'$, then $\cSem c {\rho} (0) = \tau$ and $\cSem
  c {\rho} (i+1) = \cSem{c'}{\envAdv{\rho}}(i)$ for all $i \in \nats$,
  where $\envAdv{\rho} (i) = \rho(i+1)$. If $\cSem c \rho (0) = \tau$
  then there is some $c'$ with $c \cRed\rho\tau c'$.
\end{theorem}


\section{Future Work}
\label{sec:disc-future-work}

For future work we plan to implement and certify more extensive
analyses and transformations, e.g.\ scenario generation and
``zooming'' (changing the granularity of time). Moreover, an important
goal is to generate from a contract efficient code to calculate its
payoff.


At the moment the Haskell implementation is translated by hand into
Coq definitions, which are the basis for the certification. This
approach is beneficial for rapid prototyping, but our goal is to turn
this process around and automatically extract Haskell code from the
Coq definitions.

\bibliographystyle{abbrv}
\bibliography{NWPT14Contracts}

\end{document}
