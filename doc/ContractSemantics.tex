\documentclass[xcolor=dvipsnames,11pt]{beamer} 

\mode<presentation>
{
\usetheme{Malmoe}
\useinnertheme{rounded}
\usecolortheme{beaver} 
\usecolortheme[named=Mahogany]{structure} 
}

\usepackage[utf8]{inputenc}
\usepackage[english]{babel}

\usepackage{amsmath}
\usepackage{amsfonts}
\usepackage{amssymb}

\usepackage{graphicx}

\usepackage{listings}
\usepackage{color}

%%%%%%%%%%%%%%%%%%%%%%%%%%%%%%%%%%%%%%%%%5
\newcommand{\comment}[2]{{\tiny \color{Orange}{$\spadesuit${\bf #1: }{\sf #2}$\spadesuit$}}}
%\renewcommand{\comment}[2]{}

\newcommand{\jbcomment}[1]{\comment{JB}{#1}}
\newcommand{\pbcomment}[1]{\comment{PB}{#1}}
\newcommand{\mecomment}[1]{\comment{ME}{#1}}

%%%%%%%%%%%%%%%%%%%%%%%%%%%%%%%%%%%%%%%%%5
\lstloadlanguages{ML,C}
% auto-colorisation with listings, handy for known languages...
\lstdefinestyle{smlstyle}{
      basicstyle=\ttfamily\scriptsize,
      keywordstyle=\color{violet},
      literate={->}{$\rightarrow$}1,
      stringstyle=\color{gray},
      identifierstyle=\color{blue},
      commentstyle=\color{red}\ttfamily,
      mathescape,
      language=ML
}
\lstnewenvironment{mlcode}
   {\lstset{basicstyle=\scriptsize,style=smlstyle,frame=tlrb}}
   {}
\lstset{style=smlstyle,keepspaces=true,breaklines=false}\newcommand{\cd}[1]{\lstinline$#1$}

%%%%%%%%%%%%%%%%%%%%%%%%%%%%%%%%%%%%%%%%%%
\renewcommand{\emph}[1]{\textcolor{structure!60}{#1}}

%%%%%%%%%%%%%%%%%%%%%%%%%%%%%%%%%%%%%%%%%%%%%%%%%%%%%%%%

% change here to change all
\newcommand{\ttt}[1]{\mbox{\cd{#1}}}

% smart constructors
\newcommand{\zero}{\ttt{zero}}
\newcommand{\transfOne}{\ttt{transfOne}}
\newcommand{\scale}{\ttt{scale}}
\newcommand{\transl}{\ttt{transl}}
\newcommand{\both}{\ttt{both}}
\newcommand{\ifff}{\ttt{iff}}
\newcommand{\checkWithin}{\ttt{checkWithin}}

\title{A Small Contract Language -- Semantics}
\author[Bahr,Berthold,Elsman]{Patrick Bahr, Jost Berthold, Martin Elsman}

\begin{document}

\frame[plain]{\titlepage}

\begin{frame}[fragile,t]
    \frametitle{Outline}

todo:

 - examples (vanilla option, barrier touch)                                     (mael)

 - complete complex equivalences with advancement and notion of certainty       (jost)

 - move causality to appendix and include example of non-causal contract        (jost)

 - utility examples from interpreter (cash-flows for an instrument)             (mael)

 - complete semantics slide                                                     (jost)

\begin{itemize}
\item Expressions and contracts
\item Example contracts
\item Contract equivalences
\item Contract causality
\item Contract utilities
 \begin{itemize}
  \item Finding cash flows
  \item Eliminating transfers to and from a particular party
  \item Merging parties
  \item Applying fixings
  \item Advancing a contract in time
  \end{itemize}
\item Appendix: Contract semantics
\end{itemize}
\end{frame}

\begin{frame}[fragile,t]
    \frametitle{Contract Language Features}

\textbf{Compositionality}. 
\begin{quote}
Contracts are time-relative, which makes compositionality
straightforward.
\end{quote}

\textbf{Multi-party}.
\begin{quote}
A contract specifies the contractual obligations and benefits for
multiple parties, which opens up the possibility for specifying
portfolios.
\end{quote}

\textbf{Contract management}.
\begin{quote}
Contracts can be managed; gradually, a contract reduces to the ``empty
contract''.
\end{quote}

\textbf{Contract utilities}.
\begin{quote}
Contracts can be analyzed in a variety of ways...
\end{quote}

\end{frame} 

\begin{frame}[fragile,t]
    \frametitle{Expressions and Intuitive Semantics}

\begin{mlcode}
  type 'a exp                     (* Expressions          *)
  type boolE = bool exp           (* Boolean expressions  *)
  type 'a num 
  type 'a numE = 'a num exp       (* Numeric expressions  *)
  type realE = real num exp       (*  real expressions    *)
  type intE = int num exp         (*  integer expressions *)
 
  val I   : int -> intE
  val R   : real -> realE
  val B   : bool -> boolE
  val $\oplus$  : 'a numE * 'a numE -> 'a numE  (* max, +, ... *)
  val obs : string * int -> realE
\end{mlcode}

\begin{description}
\item[I d] is the integer constant $d$.
\item[R r] is the real (i.e., double) constant $r$.
\item[$e_1$ !-! $e_2$] represents the result of subtracting the results of two subexpression.
\item[obs(s,d)] represents the value of the underlying ``s'' in $d$ days.
\end{description}

\end{frame}

\begin{frame}[fragile,t]
    \frametitle{Contract Constructors and Intuitive Semantics}

\begin{mlcode}
  type cur val EUR : cur val USD : cur    (* currencies *)
  type party = string                     (* parties *)
  type contr                              (* contracts *)
  val zero        : contr             
  val transfOne   : cur * party * party -> contr
  val scale       : realE * contr -> contr
  val transl      : int * contr -> contr
  val both        : contr * contr -> contr
  val iff         : boolE * contr * contr -> contr
  val checkWithin : boolE * int * contr * contr -> contr
\end{mlcode}

\begin{description}
\item[transfOne] is a cash flow (which happens immediately)
\item[scale] scales a cash flow by an expression (of type \cd{realE})
\item[transl] postpones a contract into the \emph{future}.

    The \cd{int} argument must be positive!

\item[checkWithin] repeatedly checks a condition (of type \cd{boolE}) for a number of days.
    The \cd{int} arg. must be positive!
\end{description}
\end{frame}

\begin{frame}[fragile,t]\frametitle{Example --- Vanilla Option}

\begin{mlcode}
val equity = "Carlsberg"  (* Carlsberg stock call option *)
val maturity = 12 * 30
val ex4 =
    let val strike = 50.0
        val nominal = 1000.0
        val obs = max(R 0.0, obs(equity,0) !-! R strike)
    in scale(R nominal,
             transl(maturity,
                    scale(obs,transfOne(EUR,"you","me"))))
    end
\end{mlcode}

\textbf{Cash flows in 360 days:}

\begin{scriptsize}
\begin{verbatim}
Cashflows:
2013-12-26 Certain [you->me] EUR (1000.0*max(0.0,(Obs(Carlsberg@0)-50.0)))
\end{verbatim}
\end{scriptsize}
\end{frame}

\begin{frame}[fragile,t]\frametitle{Example --- Barrier Touch Option}
\textbf{Instrument:}
\begin{mlcode}
datatype kind = Up | Down
fun fxBarrierTouch buyer seller curSettle amount
                   (cur1,cur2) barrier kind expiry =
  let val rate = fxRate cur1 cur2
      val cond = case kind of
                    Up   => R barrier !<! obs (rate,0)
                  | Down => obs (rate,0) !<! R barrier
      val tr   = transfOne (curSettle, buyer, seller)
  in checkWithin (cond, expiry,
                  scale (R amount, tr),
                  zero)    (* pay only if barrier is hit *)
  end
\end{mlcode}

\textbf{I buy a EUR 1000 1Y EUR/USD barrier touch Up-option:}
\begin{mlcode}
val touch =
 fxBarrierTouch "me" "you" EUR 1000.0 (EUR,USD) 1.0 Up Y1
\end{mlcode}

\textbf{Contract simplifies to:}
\begin{scriptsize}
\begin{verbatim}
CheckWithin(1.0 < Obs(FX EUR/USD@0), 1y, 
            Scale(1000.0,TransfOne(EUR,me,you)), zero)
\end{verbatim}
\end{scriptsize}
\end{frame}

\begin{frame}\frametitle{Basic Contract Equivalences}{\footnotesize

\begin{columns}
\column{0.35\textwidth}
\begin{align*}
    \transl(d,\zero)             &=  \zero\\
    \scale(r,\zero)              &=  \zero\\
    \both(\zero,\zero)           &=  \zero\\
    \scale(0,c)                  &=  \zero\\
\end{align*}

\column{0.65\textwidth}
\begin{align*}
    \ifff(c,\zero,\zero)        &= \zero\\[0.5ex]
    \ifff(T,c_1,c_2)            &=  c_1\\
    \ifff(F,c_1,c_2)            &=  c_2\\[0.5ex]
    \checkWithin(b,0,c_1,c_2)   &= \ifff(b,c_1,c_2) \\
\end{align*}
\end{columns}

\begin{align*}
    \scale(s_1,\scale(s_2,c))    &=  \scale(s_1\cdot s_2,c)\\
    \transl(d_1,\transl(d_2,c)   &=  \transl(d_1+d_2,c)\\[1ex]
    \transl(d,\both(c_1,c_2)     &=  \both(\transl(d,c_1),\transl(d,c_2))\\[1ex]
\end{align*}

}\end{frame}


\begin{frame}[t] \frametitle{Complex Contract Equivalences} {\footnotesize

\textbf{Need to explain $ exp/d$ notation at this point.}

\begin{align*}
    \transl(d,\scale(s,c))       &=  \scale(s/d,\transl(d,c))\\
    \transl(d,\scale(s,c))       &=  \scale(s,\transl(d,c))\  \mbox{if } s \mbox{ is certain}\\
%
  \transl(d,\checkWithin (b, e, c_1, c_2))  &=  \\
                             \checkWithin &(b/d, e, \transl(d,c_1), \transl(d,c_2))\\[1ex]
  \transl(d,\ifff(b,c_1,c_2))  &=  \ifff(b/d, \transl(d,c_1), \transl(d,c_2))\\
\end{align*}

}\end{frame}

\newcommand{\crule}[3]{\frac{#2}{#3}\ \mbox{\scriptsize \it #1}}

\begin{frame}[t]
    \frametitle{Contract Causality}

\emph{Causal Contracts, intention:}
{\footnotesize
\begin{quote}
 A \emph{causal contract} is a contract with the property that during all
   possible executions of the contract, a cash flow cannot depend on a
   future fixing (of an observable).
\end{quote}
Goal: Establish static bound for many contracts.
}
\smallskip
\hrule

\medskip
\emph{\textbf{Expression Causality}}

\emph{$b$-causality} $ b \vdash e$ for \emph{expressions}:
{\scriptsize $ (b \in \mathbb{Z}_0^+, \ e \in \ttt{expr0})$}

\begin{columns}
\column{0.3\textwidth}
%\jbcomment{b non-negative to use 0 as lowest value: How about $\forall b\in\mathbb{Z}: b\vdash Literal$}
\begin{align*}
\crule{(Lit)}{e \mbox{ is a literal}}{0 \vdash e}&\\[1ex]
 \crule{(Obs)}{}{max(0,i) \vdash \ttt{obs}(s,i)}&\\
\end{align*}

\column{0.3\textwidth}
\begin{align*}
&\crule{(BinOp)}{b_1 \vdash e_1\ \ b_2 \vdash e_2 }{max(b_1,b_2) \vdash e_1 \otimes e_2}\\[1ex]
&\crule{(UnOp)}{b \vdash e}{b \vdash \ominus e}\\
\end{align*}
\end{columns}
Intuitive: smallest $i$ such that no observable is observed \emph{after more than $i$ days}.

\end{frame}

\begin{frame}[t]
    \frametitle{Contract Causality (2)}

\emph{$b$-causality} $ b \vdash c$ for contracts:
{\scriptsize $ (b \in \mathbb{Z}_0^+, \ c \in \ttt{contr})$}

\begin{columns}
\column{0.3\textwidth}
\begin{align*}
\crule{(Zero)}{}{\infty \vdash \zero}&\\
\crule{(TO)}{}{0 \vdash \transfOne(C,p_1,p_2)}&\\
\crule{(TL)}{b \vdash c}{b + d \vdash \transl(d,c)}&\\
\end{align*}

\column{0.3\textwidth}
\begin{align*}
&\\
&\crule{(Sc)}{b_1 \vdash e\ \ \ b_2\vdash c\ \ \ b_1 \leq b_2 }{b_2 \vdash \scale(e,c)}\\
&\crule{(Both)}{b_1 \vdash c_1\ \ \ b_2 \vdash c_2}{min(b_1,b_2) \vdash \both(c_1,c_2)}\\
\end{align*}
\end{columns}
$$ 
\crule{(CW)}{0\vdash e\ \ \ b_1 \vdash c_1\ \ \ b_2 \vdash c_2}{min(b_1,b_2) \vdash \checkWithin(e,d,c_1,c_2)}
$$

\jbcomment{I would expect some different rules: (Zero) should use 0, not $\infty$,
(TL) should use $b-d$ because c is translated by $d$ forward, (Both) and (CW) would use $max$.
Maybe I misunderstood the directions here?

}

\end{frame}

\newcommand{\sem}[1]{[\![#1]\!]}
\newcommand{\csem}[3]{\mathcal{C}\sem{#1}#2 & = #3}

\begin{frame}
    \frametitle{Formal Semantics: Sequence of Cash Flow Sets}

An environment $\ttt{env}$ contains fixings for observables and choices.

Expression Semantics: 
\begin{align*}
  \mathcal{E} & : \ttt{boolE} \times \ttt{env} \rightarrow \ttt{bool}\\
  \mathcal{E} & : \ttt{intE} \times \ttt{env} \rightarrow \ttt{int}\\
  \mathcal{E} & : \ttt{realE} \times \ttt{env} \rightarrow \ttt{real}\\
\end{align*}

\jbcomment{explain how to move the environment $ env/d$}

\bigskip

\hrule
\bigskip

Semantics of a contracts: $ \mathcal{C} : \ttt{contr} \times \ttt{env} 
          \rightarrow \mathbb{P}(\ttt{flow})^\mathbb{N}$\\
{\footnotesize
(where $\ttt{flow}$ is a cash flow: \cd{(amount:real, c:currency, from,to:party)}
}
\jbcomment{explain flow sets as semantic domain}

\end{frame}

\begin{frame}
    \frametitle{Formal Semantics: Contracts}

Semantics of a contracts: $ \mathcal{C} : \ttt{contr} \times \ttt{env} 
          \rightarrow \mathbb{P}(\ttt{flow})^\mathbb{N}$\\
{\footnotesize
(where $\ttt{flow}$ is a cash flow: \cd{(amount:real, c:currency, from,to:party)}
}

{\footnotesize
\begin{align*}
\csem{\zero}{e}{\ttt{repeat} \emptyset}\\
\csem{\transfOne(c,p_1,p_2)}{e}{\{(1,c,p_1,p_2)\} : \ttt{repeat} \emptyset }\\
\csem{\both(c_1,c_2)}{e}{\ttt{zipWith flowMerge }\ (\mathcal{C}\sem{c_1}e)\ (\mathcal{C}\sem{c_2}e)}\\
\csem{\scale(exp,c)}{e}{}\\
\ttt{mapmap}\ &(\lambda (a,c,f,t)\mapsto(a\cdot\mathcal{E}\sem{exp}e,c,f,t))\ (\mathcal{C}\sem{c}e)\\
\csem{\transl(d,c)}{e}{\ttt{replicate }d \ \emptyset +\!\!\!+\  \mathcal{C}\sem{c}e/d}\\
\csem{\checkWithin(b,d,c_1,c_2)}{e}{
\left \{
\begin{array}{rl}
\mathcal{E}\sem{b}e: & \mathcal{C}\sem{c_1}e\\
d = 0\ \wedge\ !\mathcal{E}\sem{b}e: & \mathcal{C}\sem{c_2}e\\
d > 0\ \wedge\ !\mathcal{E}\sem{b}e:\\
\emptyset : \mathcal{C}\sem{\checkWithin&(b,d\!-\!1,c1,c_2)}e/1\\
\end{array}\right.
}
\end{align*}
}

\end{frame}

\begin{frame}[fragile,t]
    \frametitle{Contract Language Implementation in Standard ML}
\emph{Expression Datatype}
\begin{mlcode}
type party = string
                       
datatype exp0 = I of int  (* data structure for expressions *)
              | R of real (*  - numeric or boolean *)
              | ...
              | BinOp of string * exp0 * exp0
              | ...
              | Obs of string * int     (* observable *)
              | ChosenBy of party * int (* active choice *)
\end{mlcode}

\emph{Contract Datatype}
\begin{mlcode}
datatype contr =
       Zero
     | TransfOne of cur * party * party
     | Scale of exp0 * contr
     | Transl of int * contr
     | Both of contr * contr
     | If of exp0 * contr * contr
     | CheckWithin of exp0 * int * contr * contr
\end{mlcode}
%datatype contr =
%       Zero
%     | TransfOne of cur * party * party (* immediate cash flow *)
%     | Scale of exp0 * contr            (* scaling by an expr. *)
%     | Transl of int * contr            (* postpone to later *)
%     | Both of contr * contr            (* combine two contracts *)
%     | If of exp0 * contr * contr       (* choice between two contracts *)
%     | CheckWithin of exp0 * int * contr * contr
%               (* monitoring of a condition (expr.) within a duration 
%                  if true within time: first contract, otherwise second *)
%
\end{frame}



\end{document}
