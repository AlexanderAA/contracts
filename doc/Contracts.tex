\documentclass[xcolor=dvipsnames,11pt]{beamer} 

\mode<presentation>
{
\usetheme{Malmoe}
\useinnertheme{rounded}
\usecolortheme{beaver} 
\usecolortheme[named=Mahogany]{structure} 

\setbeamertemplate{navigation symbols}{}%remove navigation symbols
}

\usepackage[utf8]{inputenc}
\usepackage[english]{babel}

\usepackage{amsmath}
\usepackage{amsfonts}
\usepackage{amssymb}

\usepackage{graphicx}

% \usepackage[dvipsnames]{color} loaded by beamer, see above

\usepackage{listings}

%%%%%%%%%%%%%%%%%%%%%%%%%%%%%%%%%%%%%%%%%5
\newcommand{\comment}[2]{{\tiny \color{Orange}{$\spadesuit${\bf #1: }{\sf #2}$\spadesuit$}}}
%\renewcommand{\comment}[2]{}

\newcommand{\jbcomment}[1]{\comment{JB}{#1}}
\newcommand{\pbcomment}[1]{\comment{PB}{#1}}
\newcommand{\mecomment}[1]{\comment{ME}{#1}}

%%%%%%%%%%%%%%%%%%%%%%%%%%%%%%%%%%%%%%%%%5
\lstloadlanguages{Haskell,C}
% auto-colorisation with listings, handy for known languages...
\lstdefinestyle{hsstyle}{
  language=Haskell,
  basicstyle=\scriptsize\ttfamily,
  commentstyle=\it\color{Green},
  stringstyle=\mdseries\rmfamily\color{Gray},
  showspaces=false,
  showstringspaces=false,
  keywordstyle=\bfseries\rmfamily\color{Violet},
  morecomment=[l]{--},
  morecomment=[s]{\{-}{-\}},
  literate={+}{{$+$}}1 {/}{{$/$}}1 {*}{{$*$}}1 {=}{{$=$}}1 {>}{{$>$}}1 {<}{{$<$}}1 
           {\\}{{$\lambda$}}1
           {\\\\}{{\char`\\\char`\\}}1 {|}{{$\mid$}}1
           {->}{{$\rightarrow$}}2 
           {\\n}{{\textbackslash n}}2 
           {>=}{{$\geq$}}2 {<-}{{$\leftarrow$}}2
           {<=}{{$\leq$}}2 {=>}{{$\Rightarrow$}}2
           {\ .}{{$\circ$}}2 {\ .\ }{{\ $\circ$\ }}2
           {>>}{{>>}}2 {>>=}{{>>=}}2
  ,    
%  columns=flexible,
  emphstyle={\bf},
  mathescape,
}
\lstnewenvironment{hscode}
   {\lstset{basicstyle=\scriptsize,style=hsstyle,frame=tlrb}}
   {}
\lstnewenvironment{hscodesmall}
   {\lstset{basicstyle=\tiny,style=hsstyle,frame=tlrb}}
   {}
\lstset{style=hsstyle,keepspaces=true,breaklines=false}\newcommand{\cd}[1]{\lstinline$#1$}

%%%%%%%%%%%%%%%%%%%%%%%%%%%%%%%%%%%%%%%%%%
\renewcommand{\emph}[1]{\textcolor{structure!90}{#1}}

%%%%%%%%%%%%%%%%%%%%%%%%%%%%%%%%%%%%%%%%%%%%%%%%%%%%%%%%

% change here to change all
\newcommand{\ttt}[1]{\mbox{\cd{#1}}}

% smart constructors
\newcommand{\zero}{\ttt{zero}}
\newcommand{\transfOne}{\ttt{transfOne}}
\newcommand{\scale}{\ttt{scale}}
\newcommand{\transl}{\ttt{transl}}
\newcommand{\both}{\ttt{both}}
\newcommand{\ifff}{\ttt{iff}}
\newcommand{\checkWithin}{\ttt{checkWithin}}

\title{A Small Multi-Party Contract Language}
\author[Bahr,Berthold,Elsman,Henglein]{Patrick Bahr, Jost Berthold, Martin Elsman, Fritz Henglein}

\begin{document}

\frame[plain]{\titlepage}

\frame{\jbcomment{gathering one single big presentation to scavenge later}}

\begin{frame}[fragile,t]
    \frametitle{Outline}

{\footnotesize
\tableofcontents
}
\end{frame}

\begin{frame}[fragile,t]
    \frametitle{Contract Language Goals}

\textbf{Compositionality}. 
\begin{quote}
Contracts are time-relative, which makes compositionality
straightforward.
\end{quote}

\textbf{Multi-party}.
\begin{quote}
A contract specifies the contractual obligations and opportunities for
multiple parties, which opens up the possibility for specifying
portfolios.
\end{quote}

\textbf{Contract management}.
\begin{quote}
Contracts can be managed; gradually, a contract reduces to the ``empty
contract''.
\end{quote}

\textbf{Contract utilities}.
\begin{quote}
Contracts can be analyzed in a variety of ways...
\end{quote}

\end{frame} 

\begin{frame}[fragile,t]
    \frametitle{Expressions and Intuitive Semantics}

\begin{hscode}
data Expr a where
    I :: Int    -> Expr Int    -- Int
    R :: Double -> Expr Double -- Double
    B :: Bool   -> Expr Bool   -- Bool
    V :: Var    -> Expr a      -- Variable
-- | arithmetic operations: + - * / max min
$\oplus$ :: Num a => Expr a -> Expr a -> Expr a
-- | logical operations: < = ! |
!<! :: Ord a => Expr a -> Expr a -> Expr Bool
!=! :: Eq a  => Expr a -> Expr a -> Expr Bool
not :: Expr Bool -> Expr Bool
!|! :: Expr Bool -> Expr Bool -> Expr Bool

type RExpr = Expr Double; type BExpr = Expr Bool
\end{hscode}

\begin{description}
\item[I d] is the integer constant $d$.
\item[R r] is the real (i.e., double) constant $r$.
\item[$e_1 \oplus e_2$] is transformed automatically (ad-hoc polymorphism)
\item[$e_1 !\!\!<! e_2$] compares embedded values (includes evaluation)
\end{description}

\end{frame}

\begin{frame}[fragile,t]
    \frametitle{Expressions and Intuitive Semantics (2)}

\begin{hscode}
data Expr a where
    -- continued...
    -- Observables and choice
    obs :: (String, Int) -> Expr Double    -- observable
    chosenBy :: (String, Int) -> Expr Bool -- choice

    -- Accumulator. acc(f,i,a) := f/i(...(f/2(f/1(a))))
    acc :: (Expr a -> Expr a) -> Int -> Expr a -> Expr a

    -- Pairs
    pair :: Expr a -> Expr b -> Expr (a,b)
    fst :: Expr (a,b) -> Expr a
    snd :: Expr (a,b) -> Expr b
\end{hscode}

\begin{description}
\item[obs(s,d)] represents the value of the underlying ``s'' in $d$ days.
\item[chosenBy(s,d)] represents a choice made by party ``s'' in $d$ days.
\item[acc(f,i,a)] accumulates a value for $i$ days using $f$, starting with $a$
\end{description}

\end{frame}

\begin{frame}[fragile,t]
    \frametitle{Contract Constructors and Intuitive Semantics}

\begin{hscode}
  data Currency = EUR | USD | SEK... -- names (assets)
  type Party = String       -- parties
  data Contr -- internal    -- contracts
  zero        :: Contr             
  transfOne   :: Cur -> Party -> Party -> Contr
  scale       :: RExpr -> Contr -> Contr
  transl      :: Int -> Contr -> Contr
  both        :: Contr -> Contr -> Contr
  iff         :: BExpr -> Contr -> Contr -> Contr
  checkWithin :: BExpr -> Int -> Contr -> Contr -> Contr
\end{hscode}

\begin{description}
\item[transfOne] is a cash flow (which happens immediately)
\item[scale] scales a cash flow by an expression (of type \cd{rexp})
\item[transl] postpones a contract into the \emph{future}.

    The \cd{int} argument must be positive!

\item[checkWithin] repeatedly checks a condition (of type \cd{bexp}) for a number of days.
    The \cd{int} arg. must be positive!
\end{description}
\end{frame}

\begin{frame}[fragile,t]
    \frametitle{Examples}

\textbf{Vanilla option on Carlsberg stock}
\begin{hscode}
callOption = scale (R nominal)
                (transl maturity
                    (scale carlsb (transfOne EUR "you" "me"))))
  where maturity = oneYear -- == 365, using ACT/365
        strike  = 50.0
        nominal = 1000.0
        carlsb  = maxx 0.0 (obs ("Carlsberg",0) - strike)
\end{hscode}

\textbf{FX barrier touch option on EUR/USD rate}
\begin{hscode}
barriertouch = checkWithin (1.0 !<! obs ("FX EUR/USD",0))
                  oneYear
                  (scale 1000.0 (transfOne EUR "me" "you"))
                  zero
\end{hscode}

\begin{itemize}
\item Note: \emph{observables} - anything, \emph{transfers} - defined assets (type)
\item Multiparty: seller and buyer perspective only suggested by "you" and "me" strings
\end{itemize}

\end{frame}

\begin{frame}[fragile,t]\frametitle{Example --- Barrier Touch Option}
\textbf{Instrument:}

\jbcomment{not ported from Haskell yet, should be Contract.FXInstrument or so}
\begin{hscode}
datatype kind = Up | Down
fun fxTouch buyer seller curSettle amount
                   (cur1,cur2) barrier kind expiry =
  let val rate = fxRate cur1 cur2
      val cond = case kind of
                    Up   => R barrier !<! obs (rate,0)
                  | Down => obs (rate,0) !<! R barrier
      val tr   = transfOne (curSettle, buyer, seller)
  in checkWithin (cond, expiry,
                  scale (R amount, tr),
                  zero)    (* pay only if barrier is hit *)
  end
\end{hscode}

\textbf{I buy a EUR 1000 1Y EUR/USD barrier touch Up-option:}
\begin{hscode}
val touch = fxTouch "me" "you" EUR 1000.0 (EUR,USD) 1.0 Up y1
\end{hscode}

\textbf{Contract simplifies to:}
\begin{scriptsize}
\begin{verbatim}
CheckWithin(1.0 < Obs(FX EUR/USD@0), y1, 
            Scale(1000.0,TransfOne(EUR,me,you)), zero)
\end{verbatim}
\end{scriptsize}
\end{frame}

\begin{frame}
\frametitle{Supported Instruments}
\begin{itemize}
\item Basic swaps, fx-swaps
\item Fx-forwards
\item Vanilla options, fx-options, fx-barrier-touch options, fx-barrier-no-touch options, fx-double-barrier-in/out, fx-single-barrier-in/out
\item Asian options (not yet supported; observable average computation needed)
\item American options (supported using $\ttt{checkWithin}$)
\item Barrier options with fixed maturity (not yet supported)
\end{itemize}
\end{frame}

\begin{frame}
\frametitle{Simple contract transformation utilities}

A number of simple portfolio management operations can be implemented easily,
and keep the contract completely symbolic:

\begin{itemize}
  \item \emph{Advancing} a contract in time
  \item \emph{Eliminating} to and from \emph{a particular party}
  \item \emph{Merging parties}
\end{itemize}
\jbcomment{all these not implemented yet (Contract.Transform)}

Useful to have: a canonical normal form for each contract
\begin{itemize}
\item \emph{Normalising} a contract
\end{itemize}
\jbcomment{raises question of contract equivalence. Discuss right here, and semantics?}

\end{frame}

\begin{frame}[fragile,t]
    \frametitle{Contract Analysis: Cashflows}

\begin{hscode}
type Cashflow = (Date, Currency, Party, Party, Bool, RealE)
cashflows :: (Date, Contract) -> [ Cashflow ]
\end{hscode}

\textbf{Cash flows for Carlsberg vanilla option in 360 days:}

\begin{scriptsize}
\begin{verbatim}
Cashflows:
2013-12-26 Certain [you->me] EUR (1000.0*max(0.0,(Obs(Carlsberg@0)-50.0)))
\end{verbatim}
\end{scriptsize}

\emph{\bf Managed contract}: contract (relative dates) and start date

\begin{hscode}
cashflows :: MContract -> [ Cashflow ]
\end{hscode}

\hrulefill

\jbcomment{give an "uncertain" example, $\Rightarrow$ motivate environments}

\end{frame}

\begin{frame}[fragile,t]
    \frametitle{Observable Environments}

\begin{hscode}
-- environment: maps (observable,day) to possible value
type Env = (String, Int) -> Maybe Double
-- managed environment
data MEnv = Env Date Env -- environment and start date

emptyFrom :: Date -> MEnv -- construct empty mgd. env.

-- add values to the environment
addFixing :: (String, Date, Double) -> MEnv -> MEnv
addFixings :: (String, Date) -> [Double] -> MEnv -> MEnv

-- construct an environment from given start date and values 
fixings :: String -> Date -> [Double] -> MEnv
fixings s d vs = addFixings (s,d) vs (emptyFrom d)

-- translate an environment in time
promoteEnv :: MEnv -> Int -> MEnv
\end{hscode}

\begin{itemize}
\item Environments store information about observables

\item Contracts are \emph{symbolic} (they do not model observables)

\item Environments can/should be constructed from models
\end{itemize}

\end{frame}

\begin{frame}[fragile,t]
    \frametitle{Contract Transformation using environment information}

\textbf{Evaluating a contract (scaling, branching) as far as possible}
\begin{hscode}
-- reduce a contract, using available values in environment
simplify :: MEnv -> MContract -> MContract
\end{hscode}

\emph{Example:} Barrier option with fixings simplifies to one cashflow
\begin{scriptsize}
\begin{verbatim}
... unfinished
\end{verbatim}
\end{scriptsize}

\hrulefill

\textbf{Eliminating branches in a contract (not applying scaling)}
\begin{hscode}
-- eliminate all branches which are known to not be taken
elimBranches:: MEnv -> MContract -> MContract
\end{hscode}

(useful for scenario generation)
\end{frame}


\begin{frame}[fragile,t]
\frametitle{Example Portfolio of Touch Options}

\begin{hscodesmall}
val touchOptions = all
 [fxTouch   "C" "us" USD  400000.0 (USD,SEK) 6.90 Up   m6
 ,fxTouch   "D" "us" USD  600000.0 (USD,SEK) 6.15 Down m12
 ,fxNoTouch "A" "us" USD 1400000.0 (USD,SEK) 6.70 Up   m6
 ,fxNoTouch "B" "us" USD 1600000.0 (USD,SEK) 6.25 Down m12]

val env = fixings ("FX USD/SEK",today) 
   [6.6,6.7,6.8,6.9,6.8,6.7,6.6,6.5,6.4,6.3,6.2,6.1]

val allTouch = simplify (env,today) allTouch
val () = ppCashflows (today, allTouch)
\end{hscodesmall}

{\scriptsize
Two touch options will be triggered (barriers up 6.9, down 6.15).

Two no-touch options will be canceled (barriers up 6.7, down 6.25).
}
\vfill
\emph{Result:}
\begin{verbatim}
2014-01-03 Certain [C->us] USD 400000.0
2014-01-11 Certain [D->us] USD 600000.0
\end{verbatim}

\end{frame}

\begin{frame}[fragile]
\frametitle{Contract analysis: finding branch boundaries}

\begin{hscodesmall}
data Trigger = Trigger { underlying :: String
                       , start :: Int, end :: Int
                       , values :: [ Double ] }
-- | find all branch boundaries for a contract
branchBounds :: MContract -> [ Trigger ]
\end{hscodesmall}

\begin{itemize}
  \item Finds all observables which occur inside conditionals

      \emph{Goal:} generate tree of scenarios with fixings.
      
   \item A \cd{Trigger} is an \emph{observable} and \emph{its boundary values}
       for a certain \emph{time range} within a contract horizon.

\end{itemize}

\jbcomment{example! and explain some technicalities}

\end{frame}

\begin{frame}
    \frametitle{Contract analysis: fuzzy comparison}

\jbcomment{comparing at a semantic level, using moving average of 
    cashflows (per pair of parties)
    
Could also "defocus" a contract itself, but this is harder and involves more ad-hoc decisions    
    }

\end{frame}

\begin{frame}
    \frametitle{Future Work}

\textbf{Future work set out at some point}:
\begin{itemize}
  \item Focusing / Defocusing  (e.g., resolution: day-week-month).
  \item Pattern matching to identify standard contracts.
\end{itemize}

\textbf{More interesting now:}
\begin{itemize}
\item generate scenarios from a set of triggers
\item generate code to compute (actual) cashflows from environment information

    uncertain cashflows need to be represented as conditional ones
\end{itemize}

\end{frame}


\begin{frame}[c]
\begin{center}

{\huge APPENDIX}
\bigskip

Formal treatment and reasoning

\jbcomment{old slides, ML version, untouched}

\end{center}
\end{frame}


\begin{frame}\frametitle{Basic Contract Equivalences}{\footnotesize

\begin{columns}
\column{0.35\textwidth}
\begin{align*}
    \transl(d,\zero)             &=  \zero\\
    \scale(r,\zero)              &=  \zero\\
    \both(\zero,\zero)           &=  \zero\\
    \scale(0,c)                  &=  \zero\\
\end{align*}

\column{0.65\textwidth}
\begin{align*}
    \ifff(c,\zero,\zero)        &= \zero\\[0.5ex]
    \ifff(T,c_1,c_2)            &=  c_1\\
    \ifff(F,c_1,c_2)            &=  c_2\\[0.5ex]
    \checkWithin(b,0,c_1,c_2)   &= \ifff(b,c_1,c_2) \\
\end{align*}
\end{columns}

\begin{align*}
    \scale(s_1,\scale(s_2,c))    &=  \scale(s_1\cdot s_2,c)\\
    \transl(d_1,\transl(d_2,c)   &=  \transl(d_1+d_2,c)\\[1ex]
    \transl(d,\both(c_1,c_2)     &=  \both(\transl(d,c_1),\transl(d,c_2))\\[1ex]
\end{align*}

}\end{frame}


\begin{frame}[t] \frametitle{Complex Contract Equivalences}

\emph{\textbf{Expression promotion:}}

Expressions involving observables can be \emph{promoted} from a time later to a time earlier:
%
\hfill {\footnotesize $(e \in \ttt{iexp} \cup \ttt{rexp} \cup \ttt{bexp},\ d \in \mathbb{Z})$}

{\footnotesize 
$$ e / d = \left \{
\begin{array}{ll}
obs(s,\emph{d+i}) :& e = \emph{obs(s,i)}\\
e_1/d \oplus e_2/d :& e = e_1 \oplus e_2\\
\ldots
\end{array}
\right .$$
}

An expression is \emph{certain} if it does not depend on observables.
\medskip

\hrule
\medskip

\emph{\textbf{\ldots enables equivalences with \cd{transl}}}

{\footnotesize
\begin{align*}
    \transl(d,\scale(s,c))       &=  \scale(s/d,\transl(d,c))\\
    \transl(d,\scale(s,c))       &=  \scale(s,\transl(d,c))\  \mbox{if } s \mbox{ is certain}\\
%
  \transl(d,\checkWithin (b, e, c_1, c_2))  &=  \\
                             \checkWithin &(b/d, e, \transl(d,c_1), \transl(d,c_2))\\[1ex]
  \transl(d,\ifff(b,c_1,c_2))  &=  \ifff(b/d, \transl(d,c_1), \transl(d,c_2))\\
\end{align*}
}

\end{frame}

\newcommand{\crule}[3]{\frac{#2}{#3}\ \mbox{\scriptsize \it #1}}
\newcommand{\sem}[1]{[\![#1]\!]}
\newcommand{\csem}[3]{\mathcal{C}\sem{#1}#2 & = #3}

\begin{frame}
    \frametitle{Formal Semantics: Sequence of Cash Flow Sets}

The \emph{semantics of an expression} is (partially) defined with
respect to an environment ($\ttt{env}$), containing fixings for
observables and choices:
\vspace*{-2ex}

{\footnotesize
\begin{align*}
  \mathcal{E} & : \ttt{bexp} \times \ttt{env} \rightarrow \ttt{bool}\\
    \mathcal{E} & : \ttt{iexp} \times \ttt{env} \rightarrow \ttt{int}\\
  \mathcal{E} & : \ttt{rexp} \times \ttt{env} \rightarrow \ttt{real}
\end{align*}}
\vspace*{-2ex}

A \emph{cash flow} is a tuple \cd{(amount:real, c:currency, from:party, to:party)}.

\vfill 
The \emph{semantics of a contract} is (partially) defined, with respect
to an environment, as a series of cash flow sets:

$$ \mathcal{C} : \ttt{contr} \times \ttt{env} 
          \rightarrow \mathbb{P}(\ttt{flow})^\mathbb{N}$$

\vfill
See appendix for a full definition of contract semantics.

\end{frame}

\begin{frame}[fragile,t]
    \frametitle{Contract Causality}

 A \emph{causal contract} is a contract with the property that during all
   possible executions of the contract, a cash flow cannot depend on a
   future fixing (of an observable).

\vfill

\emph{Example of a non-causal contract:}

\begin{hscode}
iff( obs("CarlsbergDKR",2) !>! R 50.0,
                              transfOne(EUR, me, you), zero)
\end{hscode}
{\footnotesize
\begin{quote}
  If, \emph{on the day after tomorrow}, 
           the Carlsberg stock is worth more than 50~kr.,
           I give you one EUR \emph{today}.
\end{quote}}

\vfill

\emph{Goal}: Define a static contract causality analysis that, for many useful contracts,
guarantees causality (see appendix).
\end{frame}


\begin{frame}
    \frametitle{Formal Semantics: Contracts}

Expression promotion is generalised to environments: 
{\footnotesize
\begin{center}
$\ttt{env/d}$ promotes all fixings by $d$ days.
\end{center}}

Cash flows ($\ttt{flow}$): \cd{(amount:real, c:currency, from:party, to:party)}.

\vfill

Contract semantics \hfill \framebox{$ \mathcal{C} : \ttt{contr} \times \ttt{env} 
          \rightarrow \mathbb{P}(\ttt{flow})^\mathbb{N}$}

\vspace{-2ex}
{\footnotesize
\begin{align*}
\csem{\zero}{e}{\ttt{repeat}\ \emptyset}\\
\csem{\transfOne(c,p_1,p_2)}{e}{\{(1,c,p_1,p_2)\} :: \ttt{repeat}\ \emptyset }\\
\csem{\both(c_1,c_2)}{e}{\ttt{zipWith flowMerge }\ (\mathcal{C}\sem{c_1}e)\ (\mathcal{C}\sem{c_2}e)}\\
\csem{\scale(s,c)}{e}{\ttt{mapmap}\ (\lambda (a,c,f,t)\mapsto(a\cdot\mathcal{E}\sem{s}e,c,f,t))\ (\mathcal{C}\sem{c}e)}\\
\csem{\transl(d,c)}{e}{\ttt{replicate}\ d \ \emptyset +\!\!\!+\  \mathcal{C}\sem{c}(e/d)}\\
\csem{\checkWithin(b,d,c_1,c_2)}{e}{
\left \{
\begin{array}{rl}
\mathcal{C}\sem{c_1}e & \mbox{iff}\ \mathcal{E}\sem{b}e \\
\mathcal{C}\sem{c_2}e & \mbox{iff}\ d = 0\ \wedge\ !\mathcal{E}\sem{b}e  \\
\emptyset :: L & \mbox{iff}\ d > 0\ \wedge\ !\mathcal{E}\sem{b}e \ \wedge \\
& L = \mathcal{C}\sem{\checkWithin(b,d\!-\!1,c_1,c_2)}(e/1)
\end{array}\right.
}
\end{align*}
}

\end{frame}

\begin{frame}
    \frametitle{Appendix: Expression Causality}

\emph{$b$-causality} $ b \vdash e$ for \emph{expressions}:
{\scriptsize $ (b \in \mathbb{Z}_0^+, \ e \in \ttt{expr0})$}

\begin{columns}
\column{0.1\textwidth}
\emph{Axioms:}
\column{0.35\textwidth}
$$\crule{(Obs)}{}{max(0,i) \vdash \ttt{obs}(s,i)}$$
%\jbcomment{b non-negative to use 0 as lowest value: How about $\forall b\in\mathbb{Z}: b\vdash Literal$}
\column{0.2\textwidth}
$$\crule{(Lit)}{e \mbox{ is a literal}}{0 \vdash e}$$
\end{columns}


\begin{columns}
\column{0.1\textwidth}
\emph{Propagation:}
\column{0.35\textwidth}
$$\crule{(BinOp)}{b_1 \vdash e_1\ \ b_2 \vdash e_2 }{max(b_1,b_2) \vdash e_1 \otimes e_2}$$

\column{0.2\textwidth}
$$\crule{(UnOp)}{b \vdash e}{b \vdash \ominus e}$$
\end{columns}

\medskip
\hrule
\medskip

\emph{Intuition:} $i \vdash e$ indicates the smallest $i$ such that no observable in $e$ is observed \emph{after more than $i$ days}.

\end{frame}

\begin{frame}[t]
    \frametitle{Appendix: Contract Causality}

\emph{$d$-causality} $ d \vdash c$ for contracts:
{\scriptsize $ (d \in \mathbb{Z}_0^+, \ c \in \ttt{contr})$}

\begin{columns}
\column{0.3\textwidth}
\begin{align*}
\crule{(Zero)}{}{\infty \vdash \zero}&\\[1ex]
\crule{(TO)}{}{0 \vdash \transfOne(C,p_1,p_2)}&\\[1ex]
\crule{(TL)}{b \vdash c}{b + d \vdash \transl(d,c)}&\\[1ex]
\end{align*}

\column{0.3\textwidth}
\begin{align*}
&\\[1ex]
&\crule{(Sc)}{b_1 \vdash e\ \ \ b_2\vdash c\ \ \ b_1 \leq b_2 }{b_2 \vdash \scale(e,c)}\\[1ex]
&\crule{(Both)}{b_1 \vdash c_1\ \ \ b_2 \vdash c_2}{min(b_1,b_2) \vdash \both(c_1,c_2)}%\\[1ex]
\end{align*}
\end{columns}
$$ 
\crule{(CW)}{0\vdash e\ \ \ b_1 \vdash c_1\ \ \ b_2 \vdash c_2}{min(b_1,d+b_2) \vdash \checkWithin(e,d,c_1,c_2)}
$$

\medskip
\hrule
\medskip

\emph{Intuition:} $i \vdash c$ means c is causal (\textit{SC} rule) and there are no transfers before $i$ days have passed (\textit{TL} rule and using \textit{min}).

\end{frame}

\begin{frame}[fragile,t]
    \frametitle{Appendix: Contract Horizon}

\emph{$d$ is the horizon} of a contract: $ c \dashv d$:
{\scriptsize $ (c \in \ttt{contr},\ d \in \mathbb{Z}_0^+) $}

\begin{columns}
\column{0.3\textwidth}
\begin{align*}
\crule{(H-Zero)}{}{\zero \dashv 0}&\\[1ex]
\crule{(H-TO)}{}{\transfOne(C,p_1,p_2)\dashv 0}&\\[1ex]
\crule{(H-TL)}{c \dashv i }{\transl(d,c) \dashv (i + d) }&\\[1ex]
\end{align*}

\column{0.3\textwidth}
\begin{align*}
&\\[1ex]
&\crule{(H-Sc)}{b_1 \vdash e\ \ \ c \dashv b_2\ \ \ b_1 \leq b_2 }{\scale(e,c) \dashv b_2 }\\[1ex]
&\crule{(H-Both)}{c_1 \dashv b_1 \ \ \ c_2 \dashv b_2 }{\both(c_1,c_2) \dashv max(b_1,b_2) }\\[1ex]
\end{align*}
\end{columns}
$$ 
\crule{(H-CW)}{0\vdash e\ \ \ c_1 \dashv b_1 \ \ \ c_2 \dashv b_2 }{\checkWithin(e,d,c_1,c_2) \dashv (d + max(b_1,b_2)) }
$$

\medskip
\hrule
\medskip

\emph{Intuition:} $c \dashv i$ means the last transfer may happen after no more than $i$ days (\textit{H-TL} and \textit{max}). c may or may not be causal (\textit{H-SC}).

\end{frame}

\begin{frame}[fragile,t]
    \frametitle{Appendix: Contract Language Implementation}
\emph{Expression Datatype}
\begin{hscode}
type party = string                       
datatype exp0 = I of int  (* data structure for expressions *)
              | R of real (*  - numeric or boolean *)
              | ...
              | BinOp of string * exp0 * exp0
              | ...
              | Obs of string * int     (* observable *)
              | ChosenBy of party * int (* active choice *)
\end{hscode}

\emph{Contract Datatype}\footnote{Interface functions are ``smart constructor'' version of the above
constructors.}
\begin{hscode}
datatype contr =
       Zero
     | TransfOne of cur * party * party
     | Scale of exp0 * contr
     | Transl of int * contr
     | Both of contr * contr
     | If of exp0 * contr * contr
     | CheckWithin of exp0 * int * contr * contr
\end{hscode}
%datatype contr =
%       Zero
%     | TransfOne of cur * party * party (* immediate cash flow *)
%     | Scale of exp0 * contr            (* scaling by an expr. *)
%     | Transl of int * contr            (* postpone to later *)
%     | Both of contr * contr            (* combine two contracts *)
%     | If of exp0 * contr * contr       (* choice between two contracts *)
%     | CheckWithin of exp0 * int * contr * contr
%               (* monitoring of a condition (expr.) within a duration 
%                  if true within time: first contract, otherwise second *)
%


\end{frame}

\end{document}
